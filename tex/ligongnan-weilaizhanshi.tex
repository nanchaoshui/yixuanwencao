\section{未来战士}
\centerline{
	\yuesong ——试论计算机与思维的几个问题
	\blfootnote{* 一九九年六月十一日,瀚海星云。已修订个别错误。}
	\blfootnote{* 本文网址 http://bbs.ustc.edu.cn/cgi/bbscon?bn=Philosophy\&fn=M319F23B5\\\&num=2331}
}

\mbox{}

“怎么办?我爱上了一个机器人!”杰克挠了挠头。翻开近年来的科幻小说,这样的故事时有发生。曾几何时,人类一直在幻想能够造出具有和人类一样的能力或者超过人类能力的机器人从事生产劳动,以获得肉体上的解脱。(?)随着理论和技术上的发展,本世纪中叶,计算机很自然地诞生了。其超出人类想象的运算能力,使人类禁不住问自己:机器人到来的日子不远了吧?然而正当人们按捺不住心头的惊喜,随之而来的却是一种恐慌。施瓦辛格主演的科幻影片《未来战士》把这一恐慌发挥到了淋漓尽致,影片中给人类一个解决办法再简单不过:彻底毁灭一块芯片,从此,世界安宁了,未来安宁了。在心理上得到一个虚幻的安慰之后,人类不得不为飞速发展的高技术给我们带来的全新的感觉而惊叹满足。大众就这样被科学家们带入一个怪圈,在大肆宣扬计算机将给人类带来解放的同时,又时刻遭遇计算机脱离人类控制可能给人类带来灾难的恐惧。然而,这一切的背后,没有多少人意识到一个隐含的假设:计算机最终将产生思维活动。

\textbf{一、 希尔伯特第十问题与图灵机}

1900年,曾经被老师认为是弱智儿童的伟大德国数学家希尔伯特在世界数学大会上提出了二十三个数学问题,这些问题被称为希尔伯特问题。一个世纪即将过去,人类站在21世纪的门坎上,惊喜的发现大部分希尔伯特问题已经被解决。人类再一次惊诧于大脑——这个造化之奇物。

当人工智能的积极鼓动者一次又一次刺激人类神经的时候,人类却发现面临着一个尴尬的事实:希尔伯特第十问题无解。虽然这个问题表面上看起来只是一个纯粹的数学问题,然而却决定了目前计算机体系的最终能力。希尔伯特第十问题内容如下:能否通过有限步骤来判定不定方程是否存在有理整数解。推广开来,可以表述成,在给定的公里体系中,是否存在一般算法可以解决所有的数学问题。英国数学家阿伦·图灵为解答这一问题设计一个非常巧妙的思想机器——图灵机。

图灵机在计算机的发展历史具有决定性的作用,它指出了计算机(不论是电子的还是机械的)在理论层面上实现的可能。另一位计算机理论先驱冯·诺依曼提出的被称为冯·诺依曼体系,则使得电子计算机在技术层面上成为可能。现代电子计算机无一不是构架于冯·诺依曼体系之上的。如果我们更进一步考察冯·诺依曼体系,无疑会发现这一体系的基础仅仅是简单而明确的图灵机。从理论上讲现代计算机是普适图灵机。

图灵以他自己提出的图灵机考察了希尔伯特第十问题,得到的答案是否定的,这十分令人沮丧,它意味着以算法为基础现代计算机体系并不能够解决一切问题。

\textbf{二、人工智能与还原论}

还原论认为一切高级活动(生命活动,化学活动等等)都可以归结于微观世界的粒子运动。

历史上还原论无往不胜,燃素说、电的分离力、生命力等等带有浓厚神秘色彩的学说被还原论意义击破,人们打消了依靠生命力解释生命现象。依照还原论的观点,人们相信自然界的最高活动形式——意识,最终会随着技术的发展人工的实现,也就是人工智能。通常意义上的人工智能是指通过非生命形式实现。让我们考察一下一种极端想法应该是相当有益的。设想有一台超级原子操作机按照某甲在某个时刻身体中全部原子的位置原样复制一个人乙,那么这个乙肯定具有思维活动,但是乙是否会具有和甲一模一样的意识,这个问题的答案恐怕不是那么轻松能够得到的。这和克隆技术相当相近,同时比克隆技术更加极端。

不难发现上述的还原论实质上是机械还原论。然而,近代思维科学研究表明人的思维活动具有十分突出的非机械性,机械性思维只是人类思维活动的极小部分,这在艺术领域中表现尤其突出。杰出科学家的自述给我们一个极大的启示:在科学研究领域中,猜想比证明更重要,也就是我们时常提到的洞察力(在我们竭力反对神秘主义的时候,是不是也得乞灵于神秘主义?)。逻辑思维仅仅在洞察得到最终结果后的“正名”。然而,很多洞察得到的结果却是无法证明的,如:关于$x^n+y^n=z^n, n>2, n\in\mathbb{Z}, x>1, y>1, z>1$不存在正整数解的费马定理,这是少数几个没有证明就被称为定理的。事实上,不是不需要证明,而是我们没有办法至少是现在没有办法证明(逻辑的)。凯库勒的苯环表示式来得更加稀奇古怪,据称是从梦中得到的。

那么,人工智能的最终目的是什么呢?如果仅仅是逻辑演算(当然逻辑思维和逻辑演算还不是一回事,但是可以想象当现代计算机的逻辑演算能力足够大的时候,至少表观上可以表现出和逻辑思维类似的过程),现代计算机就已经足够了。然而,逻辑思维不可能也不应该是人工智能的最终目标。

\textbf{三、算法与思维}

显然,基于图灵机的现代计算机远远达不到人工智能的最终的要求,这就意味着现代计算机系统根本不可能产生意识活动。尽管在现实当中,现代计算机在交谈、写作甚至一向被认为是智力的最高表现的下棋等方面取得了令人瞠目结舌的效果,这些效果仍然按照人类为其编制的程序以其惊人的逻辑演算能力获得。在深蓝计算机与国际象棋大师卡斯帕罗夫之间国际象棋大战中,可以清楚的看到,编制一个好的计算机程序(算法的)是至关重要的。

现代计算机和大脑的关键区别在于大脑是由数量多达上亿的神经细胞通过突触和神经纤维连接在一起的网络系统(非固定连接),每个神经细胞本身就是一个类似于现代计算机的单元。基于这样的认识,神经网络计算机成为当前计算机科学领域中最为尖端的课题。随着计算机网络的发展,部分计算机科学家提出通过Internet连接在一起全世界成千上万台计算机(从深蓝这样的巨型机到我们家用的微型机)是可能实现意识现象的。虽然Internet已经显示出其不可估量的巨大能力,通过分布计算仅用了几个月的时间就成功地解开了一个理论上就是象深蓝这样的计算机也要计算上千年才能解开的密码,而且仅仅采用了穷举法。这里“量变到质变”再一次表明它的正确性。但是人们似乎忘记了人类的大脑根本不是算法的这一事实,无论怎样,我们可以断言:只要这种计算机系统(包括网络)是基于算法的,就不可能具有思维。

\textbf{四、图灵检验与人工智能}

为了检验计算机是否具有思维活动,图灵提出了在人工智能发展史上具有重要的意义的图灵检验。图灵检验表述为:把计算机和人分别置于两个屋子里,一群人在屋子外作为检验者向屋子里的计算机和人提问,答案通过第三方严格按一同样的方式传递给检验者,如果检验者不能区分屋子里某一个是人还是计算机,那么就应该认为这台接受检验的计算机具有思维。

表面上看起来图灵检验是完美无缺的。实际上,图灵检验是不完备的。首先对于接受检验的计算机来说是不公平的,它要求计算机设法模仿人的思维特点。其次,歌德尔的中文屋子揭示了在图灵检验中计算机根本不需要思维就完全可以按照算法获得图灵检验所要求的效果。歌德尔的中文屋子是这样:在图灵检验中,检验者用中文提问,而屋子里的人根本不懂中文,但是他有一套关于中文与特定符号系统的映射集,因此屋子里的人可以通过这套映射集,分解问题的符号序列,再通过算法演算符号序列,将演算后得到的符号序列映射回中文,这样被检验者就根本不需要理解问题的意思而得到了问题的答案。

人工智能如果仅仅满足于达到图灵检验的要求,那么人工智能是毫无希望的。我们必须重新考察计算机与意识之间的关系。可以看到图灵检验有一个隐含的前提:计算机思维的最终发展结果是机器人要具有和人相同的意识。

\textbf{五、计算机与思维}

显然,图灵检验的前提是十分不合理的。计算机的逻辑演算能力是人类无法比拟的。物质系统是意识的唯一载体。由于构成生命的物质系统和构成计算机的物质系统在结构上巨大差别造成了这两种物质系统在本质上的差别。因此,我们完全可以想象,计算机可能的思维活动和人类之间即使不是本质上的不同也是存在着巨大差别的。这个差别不仅仅是逻辑演算能力上的巨大差别,还应该包括别的一些重大差别。那么,未来的机器人的意识应该完全不同于人类的意识。

其次,机器人一旦具有意识,它(他?)是否还能够被看作机器?我们一直在谈论我们人的人性、权利、义务,对于具有意识的机器人是否也应该具有机器人性、权利、义务。人类是否能够把他们作为劳动奴隶一样对待。人类和机器人类之间的权力分配不可避免,人类为实现使自身从劳动解脱出来的理想而制造的机器人反过来要求自己的权利,对人类来说是个莫大的讽刺。《未来战士》这部影片极度宣扬了机器人的超越人类的能力和为控制世界而表现出的残忍,但是人们忘了这场人和机器人之间的战争根本原因是机器人为了自身的权利而造反。也许我们应该把人类的概念扩大到机器人(具有意识的):人类包括两个组成(民族?看起来不是)即生物人类和机器人类。

意识是人类的专利吗?回答这个问题其实是对人类信心的挑战。如果我们充分地相信(甚至如同爱因斯坦宣称上帝不掷骰子一样)生物人类的能力,我们也不需要未来战士,同样我们应该充分地相信在未来世界里生物人类和机器人类一定能够和平共处。是的,我们同样要宣称:劳动是人的第一需要。