\section{暮鼓晨钟九华山}

\blfootnote{二〇〇七年秋。}匆匆来到九华山,匆匆离开九华山,莲花佛国清静,也只不过是人世间的偶然而已。

到达九华山脚下已经是下午五点半了,天渐渐暗了,抬头看见一轮明月从九华山后升起,暮色中九华山青黛一般,没有雄浑,多的是柔美。如果不是周围其他人声音此起彼伏,真有点人间仙境的味道。

从山脚开往九华街的大巴蜿蜒行驶在曲折的山路上,司机很娴熟的转动方向盘,车内人随着车子左摇右晃,赞叹司机的声音不觉于耳,而司机无动于衷,想来这样的赞叹对他来说一天都有十多回,习惯了。

到达九华街的时候,已经六点钟,天色完全暗了下来。九华街上灯火闪亮,街上的游人不多,想来吃饭的吃饭,住店的住店。不见僧侣的影子,应该是做晚课的时间了。街边商店香烛、法器、山货等等琳琅满目。如果不是佛门法器成列,还真以为只是一个热闹的乡间小镇。即使莲花佛国也离不了物利二字。

本以为当晚就住在九华街,第二天一早上山。随行的朋友要做法事,与拜经台的主持联系说,法事早上五点开始。那时候索道还没有运行,问了路边的店家说,步行上山需要三四个小时,这样一来最迟凌晨一点就要出发。于是决定现在步行上山。店家听说我们要步行上山,一脸惊讶。

买一张导游图,打着手电筒,顺着石阶,开始了夜上九华山的旅程。离开了还在喧闹的九华街,夜晚的九华山其实很安静。月亮还没有完全升到天顶,手电筒的光线范围很有限,除了照亮的一点山路,周边的漆黑一篇。开始的路程不算艰难,山路逐渐陡峭起来,还真有点吃力。心想,该好好锻炼锻炼了。

山路越来越陡峭,抬眼是看不到头的石阶,只有埋头一级一级地上。一路上,隐约听见寺院的钟声与梵唱。大约半个小时,来到一个山顶,查看地图,乃是回香阁。殿堂的灯火已经熄灭,只有偏房传来笑语,走近才发现是僧侣们聚集在一起看电视。僧人也不是不食人间烟火的。过了回香阁,山路一直往下,走了不久也就比较平坦了。月亮已经升高,透过树林洒下斑驳的银色。九华山标志之一的凤凰松,在月色下是另一种意境。过了凤凰松,真正开始了攀登九华山主峰天台的路程。石阶路在山间回转,一路上遇到不少大小寺院,大多数已经是人定。偶尔能听见用功僧侣的诵经和木鱼的敲击声。我已无暇顾及这些,只是希望早点到达山顶。月色中,一路景色只是朦胧。

到达拜经台的时候,已经是九点钟。如果不是中途在一个叫做竹海饭店的地方吃了一段素斋,估计八点也就能到了。庙门已经关闭,敲门,开门的是一个二十岁左右的年轻僧侣,把我们引到会客室。一位中年僧侣接待了我的朋友,约定了第二天早晨法事的内容。

早晨四点半钟,手机的闹钟响了。朋友和我赶紧起床,他是要去参加法事的,而我是不拜神佛的。来到大雄宝殿之外,殿中已经有僧人在撞晨钟,口中念着经文,殿前香炉亦已焚香在内。天上繁星点点,炉内香烟袅袅,殿中钟声悠扬,远处夜色朦胧,多少有些远离人间的味道。五点整,殿中僧人排列整齐,法鼓咚咚,木鱼铜罄,僧人诵念经文,算不得悦耳。虽然,听不懂念的是什么,平缓稳定的节奏也能使人心气平和。站在殿外看着朋友在僧人的指导下,跪拜、焚香。愿你的祈盼都能实现。法事历经一个小时方才结束。天色逐渐明亮起来。

朋友从殿中出来之后,决定登上天台,又是一段陡峭的石阶,上到天台的时候,太阳刚刚升起,映照着云霞,瑰丽无比。自然,朋友又要去焚香膜拜,而我看看四下风景。山上已经有很多人了,巨大的香炉里已经满是燃着和燃尽的佛香。站在天台四处看去,倒也有一览众山小的气势。

七点钟,我们回到拜经台,朋友去见了主持和尚。离开拜经台,我们决定前往十王峰,围绕九华山的几个主峰转一圈。从拜经台到十王峰的山路,明显与其他山路不同。路上堆积着落叶,看来很少有人经过了。路边散落着五颜六色的纸片,仔细一看,原来是藏传佛教中的风马,看来有藏人来此转山,达到十王峰顶的时候,更证实了我的猜测,峰顶的松树间缠绕着许多风马旗。

虔诚的转山人祈盼这些风马旗上的颂愿能随着高山的风飘至佛前,真是叹服。象我这样空手上九华山都已经觉得不堪,而虔诚转山人背负这些沉重的祈盼来到山上。转过十王峰却发现又来到了天台,已经累得不行了,转山头的计划只好取消,于是乘坐索道下山。索道只需要十多分钟就达到凤凰松所在的附近。而昨天晚上却花了将近两个小时才走完。

下了缆车,我们顺着石阶路返回九华街。山间错落着一些房屋,如果不是屋前的香炉表明这是寺院的话,和民居没有两样。

一路走过,忽然遇着一座庵堂,上写“心愿庵”,朋友步入庵内,跪拜不已。一位老师太从偏房走到堂前,诵念吉祥语句,执椎敲了三下铜罄。我那位朋友紧闭双目一言不发,我知道他在佛前说了他的心愿,愿佛祖保佑……

匆匆来到九华山,匆匆离开九华山,莲花佛国清静,也只不过是人世间的偶然而已。