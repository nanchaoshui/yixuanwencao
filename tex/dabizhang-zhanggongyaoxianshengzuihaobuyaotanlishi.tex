\section{张功耀先生最好不要谈历史}
\blfootnote{* 2007年6月27日发表于《新语丝》,申明:本人对本文全部文字负责,与方舟子以及新语丝网站无关。}\blfootnote{* 本文网址 http://xys.org/xys/ebooks/others/science/dajia8/zhongyi1043.txt}张功耀先生的科学哲学功底如何,不敢评论,笔者一不是科学哲学出身,二对科
学哲学也不甚在意。虽然笔者理科出身,对于历史,本人还是敢说一说的,好歹
也读了不少历史典籍。

看了张先生几篇涉及历史的文章,深以为张先生最好于历史闭口不谈,“藏
拙”一事在古人亦是美谈。何以言之?张先生最近的一篇涉及历史的文章《中国
人民努力摆脱中医困扰的一些历史线索》
\footnote{http://xys.org/xys/ebooks/others/science/dajia8/zhongyi1035.txt}
(以下简称《中》文)又来生拉硬扯。

看了张先生的《中》文,我要请教张先生的第一个问题是康熙皇帝在位期间
属于“清朝后期”?清入关以后一共十二位皇帝,入关后十帝。入关后,顺治是
第一个,康熙是第二个。清1644年入关,1911年宣统退位,连头带尾268年。康
熙1661年即位,1722年驾崩,是“清朝初期”还是“清朝后期”?就算康熙是张
先生所谓的“国家领导人当中的第一个‘废医派’领袖人物”,怎么着也不是清
朝后期。

再者,康熙虽然因传教士的金鸡纳霜医治好了寒热症,却从来没有说过西医
比中医好的话,更没有发过什么诏书废了太医院,如何就说康熙是废医派了?以
笔者愚见,康熙是导致中国在固步自封的第一罪魁祸首,所谓“西学中源”说就
是康熙造的孽。康熙虽然乐于接受西洋新知识新学说,但是康熙目的不是推行新
知识新学说,是为了愚民。典型一例,任命西洋人掌管钦天监,是为了防止汉人
以及民间传习天文。康熙又要安抚汉人,要维护所谓的“华夏正统”。(虽然满
人一直被汉人视为蛮夷,满人入关之后一直宣称自己是黄帝的正统苗裔。这跟张
先生家谱宣称是黄帝苗裔有的一拼,不得视以为二)在康熙的授意下,一批臣子
宣称西方学问都是华夏自古就有,西人不过发挥皮毛。尤其以所谓的康熙布衣之
交梅文鼎最为代表。梁启超说康熙“窒塞民智”乃是不移之论,正当西方朝着科
学的曙光前进的时候,中国科学的发展恰恰是扼杀在康熙的手上。