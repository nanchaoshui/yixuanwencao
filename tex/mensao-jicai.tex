\section{荠菜}
\blfootnote{二〇〇四年。}转眼油菜花开,想着美味的青菜是没得吃了。在菜场里转了几圈,居然没有什么可以下口的东西,从菜场出来,篮子里不过一块豆腐,几个蘑菇,也是美味一格。

“老板,荠菜可要?”菜场门口一位五十来岁的妇女,面前一块塑料布,堆着一堆深绿色的植物。

荠菜?好东西来的。蹲下身,抓起一些在鼻子底下嗅嗅,清香非常。“可是野生的?”其实看看这深深浅浅、大大小小的荠菜,早就知道不会是地里种的。

“我一大早从河沿上挖来的。你闻闻,你看看。”

一会儿,篮子里多了一大捧荠菜。一路思索如何打发了这清香的荠菜。

\pskip
小时候,老师提起荠菜,就是旧社会如何贫穷,如何靠野菜充饥;父母提到荠菜,就是三年自然灾害,荠菜都吃没了。结果,自小就对荠菜没有好感。记得第一次吃荠菜,是妈妈包的荠菜饺子,原以为不会好吃,吃了之后才知道味道鲜美,他物难及。

对荠菜的认识还是在中学,我与同伴相约去郊外看早春的梅花。行进途中,看到几位老人在河边挖掘一种绿色的植物,大家好奇簇拥上去,七嘴八舌问了开来。“这个是荠菜。你们这满地都是荠菜。”仔细瞧来,一种嫩绿花白色的植物依地而生,它叶子形同锯齿,又像羽状一样分裂,清香扑鼻。于是大家不约而同地采摘起来,没有工具,就用手掐。不一会儿,手疼腿麻。四五个年轻人却没有那位老者来的利索。

\pskip
采食荠菜,古有先行。陆游有赞:“日日思归抢蕨薇,春来荠美忽忘归。”《本草纲目》中称荠菜为“护生草。”顾禄的《清嘉录》亦有:“荠菜花俗呼野菜花,因谚有三月三蚂蚁上灶山之语,三日人家皆以野菜花置灶陉上,以厌虫蚁。侵晨村童叫卖不绝。或妇女簪髻上以祈清目,俗号眼亮花。”可见,荠菜不仅有食用价值,还有药用价值,明目、利尿、解热、止血,据说常食荠菜还有美容之功效。农历二月初,村旁、地头、溪边或草地、麦田中,荠菜便露出了嫩嫩的尖,只那么一点点绿,就让寒气未消的大地充满春意和生机。荠菜长得快,仿佛昨天才露尖尖角,今天就已亭亭玉立,明天就会挺起长长的茎,顶着洁白的花。

\pskip
荠菜的吃法很多:将荠菜剁碎,用纱布滤去水分,与肉馅、适量盐、味精和在一起,可以蒸包子、包水饺、包小馄饨,清香扑鼻,口感极佳;还可以在滚水中烫一下捞起来,用香油凉拌,吃得满嘴的鲜香;与豆腐一起烧清汤,色泽一青二白清爽怡人,口感清淡舒适;与肉一起炒、煮荠菜粥等都可以,味道浓烈奔放。

而我却是喜欢荠菜豆腐圆子。荠菜开水焯过,细细切了,盛在白净瓷碗,很是好看。蘑菇加一点盐水煮开,已然柔软,细细切了,却是暗暗的灰色,煮蘑菇的水不要倒掉。豆腐加盐小火略煮,细细切了。细细切了?我没有这么好的功夫,不用切,用手捏碎就好。好了备料已经齐整,炮制开始。取大碗一只,放入豆腐,用手细细捏碎,加切碎的荠菜和蘑菇,加入芡粉、盐、糖、蛋青,顺着一个方向仔细拌匀。锅上坐油,火不要大,将豆腐泥捏成团状,下锅略炸成型。炸过的豆腐圆子下锅,加入煮蘑菇的水,加点盐、糖、料酒,小火煮透。火大了就散掉了。盛盘,汤在锅中略勾芡,浇在豆腐圆子上。于是,这荠菜豆腐圆子入了我口中,也不负放翁之美誉了。

