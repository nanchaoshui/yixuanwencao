\section{无题}
\leftskip=20mm
\noindent \\
\textbf{之一}\\
\\
夜深的时候,蝴蝶飞舞\\
耳朵里到处都是落叶\\
夏天的晚上\\
没有阳光\\
\\
你说不要看见这夜\\
你不是上帝\\
说有光\\
我给开电灯\\
\\
毕竟已经是二十一世纪\\
在城市的中心\\
有着钢筋水泥的森林\\
\\
点燃香烟\\
缥缈的是一片狼藉\\
\\
爱\\
就是我的原罪\\
这个世界没有伊甸\\
\newpage\noindent
\textbf{之二}\\
\\
在雨中放歌还是逃亡\\
考虑了这么久还是没有决定\\
天已经黑暗了\\
月亮再也不会升起\\
诸神在黄昏时分安息了\\
永远地安息在神圣殿堂里\\
\\
火,是谁点燃\\
诸神安息的神圣殿堂里\\
还会有谁\\
啊,是诗人\\
诗人还没有死\\
诸神的葬礼开始了\\
没有其他参观者\\
只有诗人\\
\\
天还会黑暗吗\\
神圣殿堂的大火吞噬诸神的尸体\\
烧红了天的尽头\\
诗人啊\\
你在什么地方\\
葬礼还没有结束吗\\
\\
你怎么还没有回来\\
天上面\\
诸神的尸体燃烧着熊熊火焰\\
天下面\\
冰冷的雨\\
是谁的眼泪\\
还有谁会哭泣\\
诸神都已经死去\\
\\
是你吗\\
诗人\\
你还会流泪吗\\
大火没有烤干你的眼泪\\
让它流到了地上\\
你的眼泪\\
还会润泽这片土地吗\\
还是有太多的忧伤\\
等待万物的安息\\
\\
\textbf{之三}\\
\\
你\\
出现\\
让我在\\
诚恐中承受另一种目光的绞杀\\
而我也知道\\
夏日无风的日子\\
该用某种东西来结束\\
这摇弋的渴望\\
\\
我心\\
缠绵已久\\
你,遥遥而立\\
如幽兰\\
守望在晓风残月的杨柳之岸\\
一任冷雨扑打着窗棂\\
于孤寂的夜晚\\
照彻着温暖和透明\\
\\
我们可以用最通俗的语言阐释诗歌\\
阐释玫瑰的奥意\\
只是灵魂的脆弱\\
让任何跋山涉水漂洋过海投来的目光\\
都被肢解得体无完肤\\
\\
月的朦胧\\
依然弥漫在这多情的季节\\
我却在寻找柳絮飘飞的家园\\
无法抛却的红尘如梦\\
无法企盼的柔情似水\\
\\
选择诗歌吗\\
一任我放歌而去\\
还是选择逃往\\
风雨中的怆惶\\
\\
或者还有希望\\
\newpage\noindent
\textbf{之四}\\
\\
我不想用金色的锁链\\
圈起一片金色的池塘\\
\\
在风中的呼喊\\
哑然\\
  无声\\
你走过了无数的院子\\
来到的不是我的角落\\
\\
呼喊的号角\\
在哪里响起,你\\
不曾知道\\
\\
还有泪吗\\
\\
走吧,走吧\\
路不会因为你\\
  更不会因为我\\
  而终止在黎明的晨曦\\
走吧,走吧\\
看那远方的风帆\\
  不是你的旗帜\\
\\
夏夜的月光\\
悄然\\
我知道那不是我的世界\\
也不是\\
  你的世界\\
你的世界没有我的角落\\
我的世界没有你的欢乐\\
\\
风起来的时候\\
我仿佛看到你的翅膀\\
  翩跹的秋的日光\\
你是否看见天外那一缕淡淡的金色\\
\\
去了,去了\\
去了那无尽的天国\\
\\
  你不会看到我的忧愁\\
\\
我不会看到你的期盼\\
\\
就这样,渐行渐远\\
\\
天门中断楚江开\\
  是谁在高歌\\
  碧水东流至此回\\
酒\\
没有酒的日子\\
怎会有诗的年代\\
  孤帆远影碧空尽\\
唱罢,唱罢\\
唯见长江天际流\\
  飘走\\
\\
今天,谁的歌声\\
荡漾着酒的释怀\\
\\
你?\\
我?\\
\\
不是\\
\\
结束了\\
睡吧\\
洗洗睡了\\
\\
乖\\
\\
\textbf{之五}\\
\\
初夏的夜晚\\
有许多细语\\
不能对你讲,那是我的秘密\\
\\
秘密的流言\\
在初夏的夜晚慢慢\\
慢慢\\
慢慢\\
开了花\\
\\
花开了\\
是否那样美丽,你那流言的花朵\\
\\
醉了\\
困了,睡觉\\
应该睡觉\\
\\
今天雨下得很大\\
明天\\
是否还会下\\
你告诉我\\
我告诉你?\\
\\
呢喃\\
去了\\
去了\\
去了\\
\\
我去了\\
阎王殿前,不是那么很好说话\\
投胎之后,不会是你的……
