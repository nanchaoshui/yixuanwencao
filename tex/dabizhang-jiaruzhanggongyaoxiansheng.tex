\section{假如张功耀先生……}
\blfootnote{* 2007年7月16日发表于《新语丝》,申明:本人对本文全部文字负责,与方舟子以及新语丝网站无关。}\blfootnote{* 本文网址 http://xys.org/xys/ebooks/others/science/dajia8/zhongyi1105.txt}假如张功耀先生扎实一点,仔细研读经典,搞清楚历史文献的真实背景和含义;

假如张功耀先生认真一点,不把以所谓“情理之中”推断当作事实;

假如张功耀先生大方一点,坦然接受批评;

那么对于张功耀先生的文章,我还能说什么?我又能写什么?

\mbox{}

我早就说过:“中医是否应该告别,可以讨论。”但是象张功耀先生这样生拉硬扯算什么?乱解经典又算什么?别人批评到底当没当作一会事情?去年“历法”讨论当中,张功耀先生除了想当然地为自己的观点辩护以外,接受过一点别人的意见吗?

《孔子不信巫医考》的文章,我也撰文指出过其中的错误,张功耀没有看到?还是不屑一顾?同样的错误再次出现,而且还多了一个错误。护中医的拉历史拉文献,难道废中医的也就必须拉历史拉文献?在新语丝上,历史文献有功底的人应该不少,怎么不见别人拉上作古的人?就算拉也要拉对人啊!

说实话,对于张功耀先生依据“科学哲学”来讨论中医的文章,很有理论水平。即使不能说滴水不漏,也是见到多年学术功夫的。这不是很好吗?干什么非要在自己并不擅长的方面搅来搅去?

也许,张功耀先生果真有点“历史癖”,把个追溯到黄帝这个子虚乌有人物的家谱堂而皇之的挂在自己的博客上,下面八个大字:“报本祖宗,鞠躬尽瘁”,不知所云。揣摩起来大概的意思是:报告本人的祖宗,我将鞠躬尽瘁”。但是“本人的祖宗”可以简称为“本祖宗”的话,那么是不是还可以说:本父、本母、本兄、本姊?进一步“本官”啥意思?是:“本人的官长”?这有很意思吗?张功耀先生还兴致勃勃地讨论炎帝、黄帝该不该一块祭祀(《炎黄之祭源流述略》),这篇妙文,同样拥挤着一堆对历史的误读和误解!这很有意思吗?

对于张功耀先生在中医方面的观点,我从来没有评论过。对于中医、现代医学,自知没有水平参与讨论。我参与新语丝的讨论是从张功耀先生提出“废阴历”开始的,而且我也只是对张功耀先生的在历史、文献方面的错误提出看法。

顺带说一句“文科傻妞”,如果这是侮辱性的语言,那么新语丝上面发过文章的人,有几个可能又有吃官司的希望了!