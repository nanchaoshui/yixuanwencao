\section{张功耀的家谱不一定荒唐?}

【括号内宋体为笔者的文字,括号外仿宋体为原作文字。申明:本人对括号内全部文字完全负责,与方舟子以及新语丝网站无关。】\blfootnote{* 2006年10月9日发表于《新语丝》。仅作排版调整。}
\blfootnote{* 本文网址 http://xys.org/xys/ebooks/others/science/dajia7/jiapu3.txt}


【拙文《家谱的笑话》在新语丝刊布之后,阮宗光先生作文
\footnote{阮宗光先生原文网址 http://xys.org/xys/ebooks/others/science/dajia7/jiapu2.txt}
说:张功耀的家谱不一定荒唐。来看看阮宗光先生的理由,谈谈我的看法。】

\fangsong
张功耀的家谱准确机会不高,但也不一定荒唐;
\normalfont【“准确机会不高”这个概念用得真是太妙了。只要这个机会不为零,哪怕是无穷小,理论上也是有准确的可能,所以不一定荒唐。这样文字游戏,恐怕不是阮先生的意思吧。阮先生是否应该解说一下,您在这里用“准确机会不高”真实意思。】
\fangsong
他把很多史上姓张的名人算作自己祖先,还是有顺线而上的方法。
\normalfont
【“算作自己的祖先”,呵呵,家谱上,前后代直接血缘关系是家谱的能否得以成立的一个必要条件,不是想算就算得上的。顺线而上,也是要有线才上得去的。】

\fangsong
张孟谈,是赵襄子的手下;张老,是晋赵武的顾问;春秋战国时代常有一国贵族为某些原因搬去邻国谋生的事,比如孔子的祖先是宋人;张孟谈的子或孙搬去韩国为相,不一定要是韩王族。
\normalfont
【没问题,阮先生提出这点合情合理,可以商榷。问题是,谁能证明张孟谈、张老是张良的先人?就算张孟谈、张老是张良的先人,也不能证明张功耀家谱的真实性。典籍中第一个可考的张姓之人是张仲,周宣王时人。《诗经·小雅·六月》:“侯谁在矣,张仲孝友”,注曰:“张仲,吉甫友也。善父母曰孝,善兄弟曰友。此言吉甫燕饮喜乐……而孝友之张仲在焉”。张仲之子记载在那部先秦典籍当中?张功耀的家谱中从挥之后到张仲一共五十余代又见在那部典籍上?无凭无据,拿来做祖先,怕是要认错了祖宗。】

\fangsong
钱进指出黄帝25子中没有张挥;其实当时也没有姓氏这个概念,有的是族这个概念,
\normalfont
【阮先生知一不知二。上古姓、氏是两个概念。《国语·周语》:“姓者,生也,以此为祖,令之相生,虽不及百世,而此姓不改。族者,属也,享其子孙共相连属,其旁支别属,则各自为氏。”先有姓后有氏。到周代末期,姓氏之间界限逐渐模糊,变成了一个意思。不过“享其子孙共相连属,其旁支别属”还有遗迹可循。福建南安客家人大门头上都有门额,类似“颍川衍脉”一类的文字,一则说明祖先来源,二来也有“别属”的意思在里头。农村也常见以家族所居地名冠于姓前,南头李家如何如何,村北李家又如何等等。另外,我原文说了,假使承认黄帝的存在,假使认为那些记载多少有些可信,才有前提按照姓的原本意思来讨论黄帝、挥的关系。我想阮先生不会看不出:本人不承认黄帝、挥存在的真实性。不必纠缠于此。】
\fangsong
某某人在某族长大是某族人,而某族可能是他母亲的族或父亲的族;当时也没有文字记录,周以前的族谱只能是后来补的。
\normalfont
【五六十世,平均一世算二十年,一千年左右,张先生凭什么相信这些补记就是真的,况且很多所谓补记都远远晚于周代了。阮先生又如何看呢?】

【我说张功耀先生公布的家谱是笑话在于,我都不需要去考证就知道这里面有假。假使张先生只是把自己的家谱上溯到张良的或者张良的祖父,甚是上溯到张仲,我还能马马虎虎认了(虽然,张仲时代并不会只有张仲一个姓张的,张良是不是张仲的后代也是有疑问的),考据起来是要费不少力气的。再者,除了直接血缘关系,张先生的家谱还忘记了家谱编制的一个重要的原则:“别子为祖,继别为宗,继祢者为小宗”见《礼记·丧服小记》。具体意思张先生要好好看看,阮先生不忙且有兴趣的的话,也可以去看看。】